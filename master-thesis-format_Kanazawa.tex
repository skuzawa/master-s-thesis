%\documentclass[leqno,11pt]{jarticle}
\documentclass [dvipdfmx] {jsarticle}
%\documentclass[a5j]{jarticle}

\makeatletter
\RequirePackage{marginnote,xcolor}
\setlength{\marginparsep}{5pt}
\let\temp@label\label
%\def\label#1{\marginnote{\hbox to 0pt{\scriptsize \textcolor{black!60}{#1}\hss}}\temp@label{#1}\ignorespaces}
\makeatother
\usepackage[mathscr]{eucal}
\usepackage{mathrsfs}
\usepackage{amscd}
\usepackage{amsfonts}
\usepackage{amsmath, amsthm, amssymb}
\usepackage{latexsym}
\usepackage{bm}
\usepackage[dvips]{graphics}
\usepackage{color,graphicx}
\usepackage{pdfpages}
\usepackage{algorithmic}
\usepackage{algorithm}


\numberwithin{equation}{section}

\def\tr#1{\mathord{\mathopen{{\vphantom{#1}}^t}#1}} 
\def\C{{\mathbf{C}}}%   \C == \mathbf{C}
\def\R{{\mathbf{R}}}%   \R == \mathbf{R}
\def\Q{{\mathbf{Q}}}%   \Q == \mathbf{Q}
\def\H{{\mathbf{H}}}%   \H == \mathbf{H}
\def\Z{{\mathbf{Z}}}%   \Z == \mathbf{Z}
\def\N{{\mathbf{N}}}%   \N == \mathbf{N}
\def\A{{\mathcal{A}}}%  \A == \mathcal{A}
\def\D{{\mathbf{D}}}%  \D == \mathbf{D}
\def\L{{\mathbf{L}}}%  \L == \mathbf{L}
\def\W{{\mathcal{W}}}% \W == \mathcal{W}
\def\Pa{{\mathbf{P}}}%  \Pi == \mathbf{P}
\def\Si{{\mathbf{S}}}%   \Si == \mathbf{S}

\theoremstyle{definition} %
\newtheorem{theorem}{\bf 定理}[section] % 
\newtheorem{lemma}[theorem]{\bf 補題}
\newtheorem{corollary}[theorem]{\bf 系}
\newtheorem{proposition}[theorem]{\bf 命題}
\newtheorem{fact}[theorem]{\bf 事実}
\newtheorem{conjecture}[theorem]{\bf 予想}
%
\theoremstyle{definition} % 
\newtheorem{definition}[theorem]{\bf 定義}
\newtheorem{remark}[theorem]{\bf 注意}
\newtheorem{example}[theorem]{\bf 例}
\newtheorem{notation}[theorem]{\bf 記号}
\newtheorem{assertion}[theorem]{\bf 要請}

\renewcommand{\proofname}{\bf 証明.}
\renewcommand{\algorithmicrequire}{\textbf{Input:}}
\renewcommand{\algorithmicensure}{\textbf{Output:}}


\pagestyle{plain} 

\title{Master Thesis \\ \bigskip 
修士論文 \\ \bigskip 
{\bf  遅延微分方程式を用いたプレス機制御法の開発 }\\ \bigskip 
} 
\author{九澤俊介}
\date{金沢大学大学院自然科学研究科 \\
Graduate School of Natural Science and Technology, \\
Kanazawa University}

\begin{document}
\maketitle
\thispagestyle{empty}
\setcounter{page}{-1}


\thispagestyle{empty}

\mbox{}\newpage 

\tableofcontents
\clearpage
\section{序}
遅延微分方程式とは,現時点の時間導関数が解だけでなく,前の時間における導関数にも依存する可能性のあるような微分方程式である.
\begin{equation}
\left\{\begin{array}{cc}
    X^{\prime}(t)=F\left(t, X(t), X\left(t-\tau_1\right), \cdots, X\left(t-\tau_n\right), X^{\prime}\left(t-\sigma_1\right), \cdots, X\left(t-\sigma_m\right) ;\right. & t \geq t_0 \\
    X(t)=\phi(t) ; & t \leq t_0
    \end{array}\right.
\end{equation}
単純な初期条件では初期の履歴関数$\phi(t)$を指定する必要がある.数値$\tau_i \geq 0, i=1, \ldots, n$および
$\sigma_i \geq 0, i=1, \ldots, k$は遅延,もしくは時間差と呼ばれる.
遅延は定数,関数$\tau(t)$および$t$の$\sigma(t)$(時間依存性遅延),関数$\tau(t,X(t))$および$\sigma(t,X(t))$(状態依存性遅延)である場合がある.\\
制御システムにおいてフィードバックにかかる時間遅れが指摘されており,様々な数理モデルが数多く研究されている.
時間遅れは反応,再生そしてフィードバックにかかる遅れとして数理モデルに組み込まれることが多い.
身近な例としては,シャワーの温度調節が挙げられる.熱いと感じて温度を下げるがぬるすぎて再び温度を上げる.
適温になるまで,この操作を何回か繰り返した経験を持つ人も多いのではないだろうか.これは,シャワーの温度調節機構に時間遅れがあるが
ゆえに起こることである.

また本研究は,東和精機株式会社と共同研究を行った内容である.
同社の制御システムにおける課題に遅延微分方程式を用いて数理モデルを構築し,新たな制御法の開発に取り組んだ.

本論文は次のように構成される.第2節では東和精機株式会社の問題の概要を説明する.
第3節では$\beta$コントロールによるプレス制御モデルと
それによる数値シミュレーションの結果を表す.
第4節では$\beta$コントロールによるプレス制御モデルを改良した線形コントロールによるプレス制御モデルとそれによる数値シミュレーション
結果を表す.第5節では第4節のモデルで得られた結果から見える今後の課題や展望などについて述べる.


\section{概要}
自動車の車輪の連結部に使用される金属棒をプレス機によって整形をする際,金属棒の歪んでいる方向と反対側へ指定の位置までプレスすることに
よって,真っ直ぐの状態に塑性変形させ金属棒の歪みをとりたいのだが,
現実にはその指定位置を超えてプレスしてしまう現象 (オーバーシュート) が起こる.
\begin{figure}[hbt]
    \centering
    \includegraphics[scale=0.3,keepaspectratio]{/Users/shunsuke/Desktop/master-thesis/press_image.png}
    \caption{理想の状態(左)と実際の状態(右).}
    \label{press_image}
\end{figure}
\newpage
%%\includepdf[noautoscale=true, scale=0.9, pagecommand={}]{修論図_2.pdf}

図\ref{press_image}はプレス機が金属棒をプレスするイメージ.本来は左図のように目標の押込量だけプレスしたい
が,実際には右図のようにオーバーシュートしてしまう.

オーバーシュートが生じる原因として,計測結果が歪取機に反映されるまでの遅れが挙げられる.
% \begin{figure}[hbt]
%     \centering
%     \includegraphics[scale=0.6,keepaspectratio]{/Users/shunsuke/Desktop/master-thesis/target_press.png}
%     \includegraphics[scale=0.5,keepaspectratio]{/Users/shunsuke/Desktop/master-thesis/info_image.pdf}
%     \caption{プレス機の位置イメージ}
%     \label{target_press}
% \end{figure}
\begin{figure}[hbt]
    \centering
    \includegraphics[scale=0.4,keepaspectratio]{/Users/shunsuke/Desktop/master-thesis/information.png}
    \caption{プレス機の情報伝達イメージ}
    \label{info_image}
\end{figure}

\section{$\beta$コントロールモデル}
\subsection{制御モデルについて}
目標の押込量$\ell$,制御システムの時間遅れ$\tau$,プレス速度の最大値$v_{\max}$,
最小値$v_{\min}$,指数$\beta$,時刻$t$において得られる最新の位置情報$X(t)(=x(t-\tau))$として次のプレス制御を考える:
\begin{equation}\label{eq:betamodel}
    V(X)=v_0\left(\frac{\ell-X}{\ell}\right)^\beta.
\end{equation}
ただし,初期値は$X(t)=0\ \ (0\le t\le\tau)$とする.

この制御方法に対して,与えられた$\ell,\tau,\beta$に対して,オーバーシュートが生じず,プレス時間が最も短く済む
ような,初速$v_0$を求める.

% この制御方法では,プレス機の位置が目標のプレス量に近づくにつれて減速していく.プレス機がすぐに停止できず
% オーバーシュートしてしまうため,プレス機の位置が目標のプレス量に十分近い時,低速である必要があるからである.



% \begin{equation}
%     V(X)=v_0\left(\displaystyle\frac{\min\{\ell_1,\ell-X\}}{\min\{\ell_1,\ell\}}\right)^\beta.
% \end{equation}
% $X$が十分に$\ell$に近づけば停止させる.



% 指数$\beta$に対し,
% \begin{equation}
%     G:=1-0.58\times\beta^{-0.79},\ \ell_1:=\frac{\tau v_{\max}}{G},\ v_0:=\min\left\{\displaystyle\frac{\ell G}{\tau},v_{\max}\right\}
% \end{equation}
% とおく.次のようにプレス速度$V(X)$を設定する.


\subsection{微分方程式}
$t\in\mathbb{R},x(t)\colon\mathbb{R}\rightarrow\mathbb{R}$と定数$\ell,\tau,\beta,v_0\in\mathbb{R}$
に対して\eqref{eq:betamodel}は次のように書き換えられる.
\begin{equation}\label{betafomula}\begin{cases}
    \displaystyle\frac{dx}{dt}(t)=v_0\left(\frac{\ell-x(t-\tau)}{\ell}\right)^\beta &(t>\tau),\\
    x(t)=v_0 t &(0\le t \le \tau).
\end{cases}\end{equation}


\subsubsection{正規化}
$s,w(s),u_0$を
\begin{equation}
    s:=\frac{t}{\tau},\ w(s):=\frac{x(\tau s)}{\ell},\ u_0:=\frac{\tau v_0}{\ell},
\end{equation}
と定めると,\eqref{betafomula}は次のように書ける.
\begin{equation}\label{betafomula2}\begin{cases}
    \displaystyle\frac{dw}{ds}(s)=u_0(1-w(s-1))^\beta &(s>1),\\
    w(s)=u_0 s &(0\le s \le 1).
\end{cases}\end{equation}


\subsection{数値計算}
\subsubsection{離散化}
$2\le K\in\mathbb{N}$として,時間刻み幅を$\Delta t:=1/K$とおく.

初期条件を$W_n=u_0n\Delta t$ ($n=0,1,...,K)$として,$w_n $ ($n=K+1,K+2,...$)
を次で求める.
\begin{equation}
    \displaystyle\frac{w_{n+1}-w_n}{\Delta t}=u_0(1-w_{n-K})^\beta.
\end{equation}
\subsubsection{アルゴリズム}
\begin{figure}[h]
    \begin{algorithm}[H]
        \caption{$\beta$コントロール}
        \label{alg1}
        \begin{algorithmic}[1]    
        \REQUIRE $2\le K \textless N\in\mathbb{N}$
        \ENSURE $w[N]$
        \FOR{$n=1,\ldots,N$}
        \IF{$n \le K$}
        \STATE $y \leftarrow 0$
        \ELSE
        \STATE $y \leftarrow w[n-K]$
        \ENDIF
        \IF{$y>L$(オーバーシュートしているか.)}
        \STATE $u \leftarrow 0$
        \ELSE
        \STATE $u \leftarrow u_0(1-y)^\beta$
        \ENDIF
        \STATE $w[n+1]\leftarrow w[n]+u\times ds$
        \ENDFOR
        \end{algorithmic}
    \end{algorithm}
\end{figure}
\subsubsection{数値計算結果}
($\beta$を固定して$u_0$をパラメータとした結果)
\section{線形コントロールによる分析}
従来の$\beta$コントロールでは以下のような課題があった.
\begin{itemize}
    \item プレス機をシャフトに衝突してから制御し始めていたため,押込量が小さいときオーバーシュートが大きくなる.
    \item 初速を$v_0(<v_{\max})$にすると,時間がかかる.
    \item 停止直前の速度が小さすぎる.
    \item 制御速度が押込量に依存しているため,オーバーシュートが安定しない.
\end{itemize}

このような課題を受け,以下のような改善点を基に線形モデルを構築した.

\begin{itemize}
    \item \eqref{betafomula}について$\beta=1$として線形コントロールすることでプレス機の速度を上げる. 
    \item 初速は常に$v_{\max}$とする.
    \item 押込量が小さいときはプレス機がシャフトに接触する前から制御を始める.
    \item 制御速度を押込量に依存させないようにする.
\end{itemize}
\eqref{betafomula}について$\beta=1$とすると,プレス速度が上がり,オーバーシュートが大きくなってしまうという問題が発生してしまう.
そのため目標の押込量の手前でプレス機を止める指令を出すことで,プレス速度を落とすことなく目標の押込量に近づけることを目指した.
目標の押込量$\ell$に対して,目標の押込量より少し短い$\ell_0$を
以下のように定義する.
\begin{equation}\label{delta_ell}
    \Delta\ell :=\delta v_{\min}\tau,\ \ell_0:=\ell-\Delta\ell
\end{equation}
ただし,$\delta$はパラメータ,$v_{\min}$は最小速度,$\tau$は時間遅れである.
目標の押込量$\ell$から$\Delta\ell$手前で止め,最小速度で時間遅れ分プレスされ,ちょうど目標の押込量で止まると考えた.

また,$\beta$コントロールではプレス機をシャフトに衝突してから制御していたが,線形コントロールでは,シャフトに衝突する前に制御を開始する.\\
$\beta$コントロールでは制御開始点$x(0)=0$としていたが,線形コントロールでは,
\begin{equation}\label{x_star}
    x_\ast:=\ell_0-\displaystyle\frac{v_{\max}-v_{\min}}{\alpha}\tau
\end{equation}
と定義し,制御開始点を$x(0)=x_\ast$と設定する.\\
$x_\ast$を設定することで,
押込量が小さい場合,プレス機がシャフトに衝突する前に制御を始めることが可能になる.
したがって線形コントロールでは\eqref{delta_ell},\eqref{x_star}で定義した$\Delta\ell , x_\ast$を用いて制御する.

\begin{figure}[hbt]
    \centering
    \includegraphics[scale=0.4,keepaspectratio]{/Users/shunsuke/Desktop/master-thesis/press_exp.pdf}
    \caption{$\beta$コントロール(左)と線形コントロール(右).}
    \label{Sample p.2}
\end{figure}
\newpage

\subsection{制御モデルについて}
目標の押込量$\ell:$,制御システムの時間遅れ$\tau:$,パラメータ$\alpha,\delta:$,
プレス速度の最大値$v_{\max}:$,プレス速度の最小値$v_{\min}$,時刻$t$で得られる最新の位置情報$X(t):$
$(=x(t-\tau))$に対し,
\begin{equation}
    \Delta\ell :=\delta v_{\min}\tau,\ \ell_0:=\ell-\Delta\ell,\ a:=\frac{\alpha}{\tau}
\end{equation}
とし,$V(X)$を次のように設定する:
\begin{equation}\label{eq:linermodel}
    V(X)=v_{\min}+a(\ell_0-X)
\end{equation}

$X$が$\ell_0$に十分近づけば停止させる.


\subsubsection{微分方程式}
\eqref{eq:linermodel}は次のように書き換えられる.
\begin{equation}\label{eq:liner}\begin{cases}
    \displaystyle\frac{dx}{dt}(t)=v_{\min}+a(\ell_0-x(t-\tau)) &(t>\tau),\\
    x(t)=v_{\max}t+x_\ast &(0\le t \le \tau)
\end{cases}\end{equation}

ただし,$x_\ast$は\eqref{x_star}で定義した位置.

\subsubsection{正規化}
次のような記号を定める.
\begin{equation}
    s:=\displaystyle\frac{t}{\tau},\ y(s):=\displaystyle\frac{x(\tau s)}{v_{min}\tau},\
    d_0:=\frac{\ell_0}{v_{\min}\tau},\ u_{\max}:=\displaystyle\frac{v_{\max}}{v_{\min}},\ 
    y_\ast:=d_0-\displaystyle\frac{(u_{\max}-1)}{\alpha}
\end{equation}
このとき,\eqref{eq:liner}は次のように書ける.
\begin{equation}\label{eq:liner2}\begin{cases}
    \displaystyle\frac{dy}{ds}(s)=1+\alpha(d_0-y(s-1)) &(s>1),\\
    y(s)=u_{\max}s+y_\ast &(0\le s \le 1).
\end{cases}\end{equation}
さらに$z(s):=d_0-y(s)$とおくと,\eqref{eq:liner2}は次のように書ける.
\begin{equation}\label{z_press}\begin{cases}
    \displaystyle\frac{dz}{ds}(s)=1-\alpha z(s-1) &(s>1),\\
    z(s)=\displaystyle\frac{1}{\alpha}(u_{\max}-1)-u_{\max}s &(0\le s\le 1).
\end{cases}\end{equation}
$z(s)$が$0$に十分近づけば停止させる.
% \eqref{z_press}は$\alpha,\ u_{\max}$に依存しており,$u_{\max}$は機械によって決まっている.\\
% したがって,最適なプレス制御となるパラメータ$\alpha$を数値計算により決定する.


\subsection{数値計算}
\subsubsection{離散化}
$2\le K\in\mathbb{N}$として,時間刻み幅を$\Delta t:=1/K$とおく.
初期条件を$z_n=(u_{\max}-1)/\alpha-u_{\max} n \Delta t$ ($n=0,1,...,K$)として,$z_n$ ($n=K+1,K+2,...$)
を次で求める.
\begin{equation}\label{z_method}
    \displaystyle\frac{z_{n+1}-z_n}{\Delta t}=1-\alpha z_{n-K}
\end{equation}
\subsubsection{アルゴリズム}
\begin{figure}[h]
    \begin{algorithm}[H]
        \caption{線形コントロール}
        \label{alg1}
        \begin{algorithmic}[1]    
        \REQUIRE $2\le K \textless N\in\mathbb{N}$
        \ENSURE $z[N]$
        \FOR{$n=1,\ldots,N$}
        \IF{$n \le K$}
        \STATE $y \leftarrow x_\ast$(位置の初期値.)
        \ELSE
        \STATE $y \leftarrow w[n-K]$($K$ステップ前の位置.)
        \ENDIF
        \IF{$y>L0$(オーバーシュートしているか.)}
        \STATE $u \leftarrow 0$
        \ELSE
        \STATE $u \leftarrow 1-\alpha \times y$
        \ENDIF
        \STATE $z[n+1]\leftarrow z[n]+u\times ds$
        \ENDFOR
        \end{algorithmic}
    \end{algorithm}
\end{figure}
\newpage
\subsubsection{数値計算結果}
\begin{figure}[h]
\begin{center}
\includegraphics[scale=0.7]{/Users/shunsuke/Desktop/master-thesis/alpha_12_8.png}
%\caption{\eqref{z_method}}
\caption{}
\end{center}
\end{figure}

\begin{table}[h]
    %\caption{\eqref{z_method}において$\alpha=0.3,0.7,1.2$で比較した値}
    \label{table:colortbl}
    \centering
    \begin{tabular}{|r|r|r|}
     \hline
     $\alpha$ & オーバーシュート$[\mu m]$ & かかった時間$[ms]$  \\
     \hline
     0.2 & -4.79 & 1972 \\
     0.4 & -1.22 & 758 \\
     0.6 & 1.51 & 460 \\
     0.8 & 3.89 & 371 \\
     1.0 & 5.50 & 329 \\
     \hline
    \end{tabular}
\end{table}
  
上図のように$\alpha$の値が小さくなるにつれ,オーバーシュートが小さくなるがプレスにかかる時間が長くなってしまう.
\newpage

\begin{figure}[h]
    \begin{center}
    \includegraphics[scale=0.7]{/Users/shunsuke/Desktop/master-thesis/delta_12_8.png}
    %\caption{\eqref{z_method}において$\delta=0.3,1.5,3.0$で比較した図}
    \caption{}
    \end{center}
\end{figure}

\begin{table}[h]
    %\caption{\eqref{z_method}において$\alpha=0.3,0.7,1.2$で比較した値}
    \label{table:colortbl2}
    \centering
    \begin{tabular}{|r|r|r|}
     \hline
     $\beta$ & オーバーシュート$[\mu m]$ & かかった時間$[ms]$  \\
     \hline
     0.3 & 4.21 & 460 \\
     1.5 & 3.01 & 460 \\
     3.0 & 1.51 & 460 \\
     10.0 & -5.48 & 460 \\
     30.0 & -15.48 & 460 \\
     \hline
    \end{tabular}
\end{table}
上図のように$\beta$の値が小さくなるにつれ,オーバーシュートは小さくなる.またかかった時間やプレスの速度は$\beta$に
依らないので変化しない.
\section{終章}
\subsection{結論}
\subsection{残された課題}
文献は\cite{Si}で表せます.
\begin{thebibliography}{9}
\bibitem{Si} A. Simoni\u{c}, The Ahlfors lemma and Picard's theorems, 
Rose-Hulman Undergrad. Math. J. {\bf 16} (2015), no. 2, 113--136. 
\bibitem{KF} 岸正倫,藤本坦孝,複素関数論,学術図書出版社 (1980).
\bibitem{Fu} 藤本坦孝,複素解析,岩波書店 (2006).


\end{thebibliography}

\end{document}
